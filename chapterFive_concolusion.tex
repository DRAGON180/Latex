\فصل{ جمع بندی و کارهای آینده}
\قسمت{ جمع بندی و کارهای آینده}
امروزه سیستم های پردازش ابری یکی از موضوعات حیاتی و مهم در زمینه فناوری اطلاعات می باشد. به کارگیری این تکنیک در کاهش هزینه ها ، کاهش زمان اجرا و .... تاثیر گذار است.مباحثی مانند توان مصرفی مراکز داده ، زمان پاسخ ، کیفیت سرویس کاربر و هزینه  ها از مباحث مهمی است که درحوزه سیستم های پردازش ابری مورد توجه زیادی قرار گرفته است.در نتیجه استفاده از راهکارهای موثر و مدیریت مناسب ماشین های مجازی و کنترل مهاجرت های رخ داده می تواند در کاهش مواردی مانند توان مصرفی، تعداد مهاجرت ها و نقض کیفیت سرویس تاثیر بگذارد.کارهای زیادی در حوزه  بهبود بهره وری انرژی و کیفیت سرویس در مراکز داده ابری صورت گرفته است.روش هایی همچون جای دهی و ترکیب پویای ماشین های مجازی در مراکز داده ابری روش های موثری برای کاهش توان مصرفی می باشد. روش های مربوط به ترکیب پویای ماشین‌های مجازی این ویژگی را فراهم می کند تا با استفاده از امکان مهاجرت ماشین های مجازی از ماشین های فیزیکی حداقلی در مراکز داده استفاده شود. 
در این پایان نامه ما سعی کردیم با  استفاده مناسب از منابع موجود ماشین های فیزیکی و جای دهی درست و مناسب و در نهایت با انتخاب ماشین مجازی مناسب به منظور مهاجرت به اهداف همچون بهبود توان مصرفی و کیفیت خدمات دست یابیم. 
در این پایان نامه روشی به منظور جای دهی اولیه ماشین‌های مجازی به همراه اعمال کنترلی در انتخاب ماشین مجازی به منظور مهاجرت در نظر گرفته شده است.روش پیشنهاد شده از طریق شبیه ساز کلودسیم مورد بررسی و ارزیابی قرار گرفته است. نتایج آزمایشات نشان می‌دهد که اعمال روش مناسب در جای دهی و کنترل کردن مهاجرت به منظور جلوگیری از مهاجرت اضافی  می تواند ما را در دست یافتن به اهدافی مانند بهبود توان و کیفیت سرویس کمک کند.
برای این منظور ، از جمله کارهایی  که در آینده بیشتر  تمایل داریم به آن ها توجه کنیم، می توانیم به تکنیک های مربوط زمانبدی که در کاهش زمان اجرای برنامه تاثیر گذار است اشاره کنیم.همچنین با اعمال پارامتر های مربوط به هزینه ها و اعمال دستگاه های خنک کننده می توانیم در کاهش هزینه ها نیز بکوشیم. 