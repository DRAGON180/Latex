% !TeX root=thesis.tex
%\فصل{بررسی و ارزیابی راه حل پیشنهادی}\برچسب{فصل‌بررسی}\صفحه‌جدید\صفحه‌پاک
\mychapter{1}{فصل چهارم: بررسی و ارزیابی راه حل پیشنهادی}\label{فصل‌بررسی}
%\chapter[فصل چهارم: بررسی و ارزیابی راه حل پیشنهادی]{بررسی و ارزیابی راه حل پیشنهادی}\label{فصل‌بررسی}
\قسمت{محیط آزمایش}
  به منظور بررسی و ارزیابی کار خود و روش مورد مقایسه ، شبیه ساز انتخاب شده کلودسیم
 \پانویس{
 \lr{CloudSim}} 

   ورژن
۳٫۰
         می باشد که یکی از ابزارهای مهم و معروف شبیه سازی در سیستم های ابری می باشد. کلودسیم یک چارچوب شبیه سازی جدید، عمومی و قابل توسعه می باشد.این ابزار به عنوان یک چارچوب شبیه سازی در در دانشگاه 
         \lr{Melbourne} 
         توسعه یافته است. امکان مدلسازی بدون لایه، شبیه سازی روی زیرساخت طراحی شده محاسبات ابری را فراهم می آورد. این ابزار پلتفرمی است که می تواند برای مدل کردن مراکز داده ،ماشین های فیزیکی ،ماشین های مجازی،  سیاست های زمانبندی و تخصیص ماشین های مجازی به میزبان های فیزیکی استفاده شود.این چارچوب یک موتور مجازی سازی را با جنبه های افزوده ای برای مدلسازی ایجاد و مدیریت موتورهای مجازی در یک مرکز داده ای ارائه می کند (تانی و آل‌عمرانی ۲۰۱۶، ۴).
%\cite{num17}.
         به منظور شبیه سازی روش خود ، محیط را ناهمگن در نظر گرفته ایم. برای این منظور، طبق مقاله پاتل
%\cite{num15}
          که به عنوان مقاله پایه در نظر گرفته شده است، ماشین های فیزیکی را در  دو حالت در نظر گرفته ایم.در حالت اول ، بهره پردازنده با
           ۱۸۶۰ 
          میلیون دستورالعمل در ثانیه
\lr{(MIPS)}
\پانویس{
\lr{Million Instructions Per Second}
}
            می باشد و در حالت دوم بهره پردازنده ماشین فیزیکی با 
          ۲۶۶۰ 
          میلیون دستورالعمل در ثانیه می باشد. مقدار حافظه
           \lr{RAM}
            ،
             ۴
           گیگابایت و پهنای باند شبکه 
           ۱
\lr{            GB/s  }
           برای هر ماشین فیزیکی در نظر گرفته ایم.ماشین های مجازی نیز دارای ویژگی های ناهمگن می باشند.برای ماشین های مجازی نیز ظرفیت پردازشی
            ۵۰۰
             ، 
             ۱۰۰۰
             ، 
             ۲۰۰۰ 
           و پهنای باند
            ۱۰۰۰۰ 
           در نظر گرفته ایم.در جدول 
(\ref{جدول۱})
            و
(\ref{جدول۲})
              مشخصات ماشین های مجازی و میزبان های فیزیکی در قالب جدول بیان شده است
           \newpage
 % !TeX root=thesis.tex
\شروع{لوح}
\تنظیم‌ازوسط
\شرح{مشخصات ماشین‌های مجازی}
\برچسب{جدول۱}
\شروع{جدول}{|c|c|c|c|c|}
\خط‌پر
‌\lr{BW & PesNumber & MIPS & Ram & VM} \\
\خط‌پر
۱۰٫۰۰۰ & 1 & 500 & 870 & 0 \\
\خط‌پر

۱۰٫۰۰۰ & 1 & ۱٫۰۰۰ & 1740 & ۱ \\
\خط‌پر
۱۰٫۰۰۰ & 1 & ۲٫۰۰۰ & 1740 & ۲ \\
\خط‌پر
\پایان{جدول}
\پایان{لوح}

 % !TeX root=thesis.tex
\شروع{لوح}
\تنظیم‌ازوسط
\شرح{مشخصات میزبان فیزیکی}
\برچسب{جدول۲}
\شروع{جدول}{|c|c|c|c|c|}
\خط‌پر
‌\lr{BW & PesNumber & MIPS & Ram & Host} \\
\خط‌پر
۱٫۰۰۰٫۰۰۰ & ۲ & ۲۶۶۰ & 4GB & 0 \\
\خط‌پر

۱٫۰۰۰٫۰۰۰ & ۲& ۱۸۶۰ & ۴GB & ۱ \\
\خط‌پر
\پایان{جدول}
\پایان{لوح}

\قسمت{نتایج مربوط به شبیه سازی}
در نمودارهای مورد آزمایش ، برای بیان کردن روش خود از واژه
 \lr{S\_S}
 \پانویس{
\lr{Suggested solution} 
}
و برای بیان روش مورد مقایسه از واژه
\lr{B\_M} 
 \پانویس{
\lr{Basic method } 
}
استفاده کرده ایم. 

برای مقایسه کار خود و روش مورد مقایسه طبق شبیه ساز کلودسیم به بررسی انرژی مصرف شده در کل اجرای برنامه و نقض کیفیت سرویس رخ داده شده که در فصل
\رجوع{فصل‌پیشنهاد} 
 آن را بررسی کردیم ، پرداخته ایم.

در نمودار
(\ref{ch1})
و
(\ref{ch2})
به بررسی توان مصرف شده با سیاست
\lr{MMT}
و
\lr{MU}
 پرداخته ایم. توان مصرفی کل مراکز داده بر حسب کیلو وات اندازه گیری می شود. به منظور مقایسه کار خود ،کار خود و مقاله پایه را با تعداد ماشین های مجازی متفاوتی که شامل ۱۰۰، ۱۵۰، ۲۰۰، ۲۵۰  می‌باشد مورد بررسی قرار داده ایم.
\صفحه‌جدید
 % !TeX root=thesis.tex
\begin{curve}
		\centering
		\SepMark{-}
\fbox{	
	\begin{tikzpicture}
\begin{axis}[
	y tick  label  style={
    /pgf/number format/1000 sep= },
	xlabel=\rl{توان مصرف شده بر حسب کیلووات},
	ylabel=\rl{تعداد ماشین‌های مجازی},
	legend style={at={(0.5,-0.3)},
	anchor=north,legend columns=-1},
	xbar=5pt,
	ymax=270,ymin=80,
	      bar width=0.9em,
	height=6cm,width=12cm,
		        nodes near coords,
]
\addplot 
	coordinates {(0.26,100) (0.33,150) 
		(0.45,200) (0.62,250)};
\addplot 
	coordinates {(0.25,100) (0.31,150)
		 (0.42,200) (0.61,250)};

\legend{ B\_M,S\_S}
\end{axis}
\end{tikzpicture}}
\caption{مقایسه مصرف انرژی با سیاست
\lr{MMT}
}
\label{ch1}
\end{curve}

 \begin{figure}

		\centering
\fbox{	
	\begin{tikzpicture}
\begin{axis}[
	y tick  label  style={
    /pgf/number format/1000 sep= },
	xlabel=\rl{توان مصرف شده بر حسب کیلووات},
	ylabel=\rl{تعداد ماشین‌های مجازی},
	legend style={at={(0.5,-0.3)},
	anchor=north,legend columns=-1},
	xbar=5pt,
	ymax=270,ymin=80,
	      bar width=0.9em,
	height=6cm,width=12cm,
	   % yticklabel={\pgfmathtruncatemacro\tick{\tick}\tick},   
	        nodes near coords,
]
\addplot 
	coordinates {
		(43,100) 
		(51,150)
		 (79,200) 
		 (100,250)
	 };
\addplot 
	coordinates {
		(45,100)
		 (52,150) 
		(81,200)
		 (104,250)
	 };
\legend{ B\_M,S\_S}
\end{axis}
\end{tikzpicture}}
\caption{توان مصرف شده بر حسب کیلووات}
\label{ch2}
\end{figure}
  همانطور که از نمودار
( \ref{ch1})
 و
( \ref{ch2})
  ملاحظه می شود، با تعداد متفاوتی از ماشین های مجازی در حالات مختلف توان مصرف شده روش پیشنهادی نسبت به کار مورد مقایسه کاهش داشته است. دلیل این کاهش در این است که ما در ابتدا زمانی که ماشین های مجازی را به میزبان های فیزیکی تخصیص دادیم سعی کردیم از منابع میزبان های فیزیکی مناسب استفاده کنیم.سعی کردیم با اعمال جای دهی مناسب در حفظ تعادل بار که در بهبود توان مصرف تاثیر گذار است  ، موثر واقع شویم.

\صفحه‌جدید 
در نمودار
	\begin{curve}

		\centering
\fbox{	
	\begin{tikzpicture}
\begin{axis}[
	y tick  label  style={
    /pgf/number format/1000 sep= },
	xlabel=\rl{تعداد مهاجرت‌های صورت گرفته},
	ylabel=\rl{تعداد ماشین‌های مجازی},
	legend style={at={(0.5,-0.3)},
	anchor=north,legend columns=-1},
	xbar=5pt,
	ymax=270,ymin=80,
	      bar width=0.9em,
	height=6cm,width=12cm,
	   % yticklabel={\pgfmathtruncatemacro\tick{\tick}\tick},   
	        nodes near coords,
]
\addplot 
	coordinates {
		(45,100)
		 (52,150) 
		(81,200)
		 (104,250)
	 };
\addplot 
	coordinates {
		(43,100) 
		(51,150)
		 (79,200) 
		 (100,250)
	 };

\legend{ B\_M,S\_S}
\end{axis}
\end{tikzpicture}}
\caption{تعداد مهاجرت‌های رخ داده با سیاست 
\lr{MMT}
 }
\label{ch3}
\end{curve}

		\begin{plot}

		\centering
\fbox{	
	\begin{tikzpicture}
\begin{axis}[
	y tick  label  style={
    /pgf/number format/1000 sep= },
	xlabel=\rl{تعداد مهاجرت‌های صورت گرفته},
	ylabel=\rl{تعداد ماشین‌های مجازی},
	legend style={at={(0.5,-0.3)},
	anchor=north,legend columns=-1},
	xbar=5pt,
	ymax=270,ymin=80,
	      bar width=0.9em,
	height=6cm,width=12cm,
	   % yticklabel={\pgfmathtruncatemacro\tick{\tick}\tick},   
	        nodes near coords,
]
\addplot 
	coordinates {
		(46,100)
		 (52,150) 
		(83,200)
		 (105,250)
	 };
\addplot 
	coordinates {
		(44,100) 
		(51,150)
		 (80,200) 
		 (101,250)
	 };

\legend{ B\_M,S\_S}
\end{axis}
\end{tikzpicture}}
\caption{تعداد مهاجرت‌های رخ داده با سیاست 
\lr{MU}
 }
\label{ch4}
\end{plot}

(\ref{ch3})
و
(\ref{ch4})
به بررسی تعداد مهاجرت های رخ داده در کل اجرای برنامه ها  پرداخته ایم.تعداد ماشین های مجازی ۱۰۰، ۱۵۰، ۲۰۰، و ۲۵۰ در نظر گرفته شده است. 

همانطور که در نمودار 
(\ref{ch3})
 ملاحظه می شود، به ازای تعداد مختلف ماشین های مجازی روش پیشنهادی بهبودی در تعداد مهاجرت های رخ داده نسبت به روش پایه داشته است.علت این بهبود در این است که با تغییراتی در سیاست‌های 
 \lr{MMT}
 و
 \lr{MU}
 و با
  انتخاب سیاست مناسب در انتخاب ماشین مجازی برای مهاجرت و جای دهی مناسب سعی کردیم در کاهش تعداد مهاجرت ها که عامل موثری در نقض کیفیت سرویس می‌باشد، تاثیر بگذاریم.
\صفحه‌جدید
% !TeX root=thesis.tex
\begin{plot}
	\noindent
		\centering
\fbox{	
	\begin{tikzpicture}
\begin{axis}[
	y tick  label  style={
    /pgf/number format/1000 sep= },
	xlabel=\rl{کیفیت سرویس نقض شده},
	ylabel=\rl{تعداد ماشین‌های مجازی},
	legend style={at={(0.5,-0.3)},
	anchor=north,legend columns=-1},
	xbar=5pt,
	ymax=270,ymin=80,
	      bar width=0.9em,
	height=6cm,width=12cm,
		        nodes near coords,
]

\addplot 
	coordinates {(0.24,100) (0.30,150) 
		(0.22,200) (0.24,250)};
		\addplot 
	coordinates {(0.24,100) (0.29,150)
		 (0.2,200) (0.23,250)};
\legend{ B\_M,S\_S}
\end{axis}
\end{tikzpicture}}
\caption{مقایسه کیفیت سرویس نقض شده در روش پیشنهادی و موردمقایسه با سیاست
\lr{MMT}
}
\label{ch5}
\end{plot}

% !TeX root=thesis.tex
\begin{plot}
	\noindent
		\centering
\fbox{	
	\begin{tikzpicture}
\begin{axis}[
	y tick  label  style={
    /pgf/number format/1000 sep= },
	xlabel=\rl{کیفیت سرویس نقض شده},
	ylabel=\rl{تعداد ماشین‌های مجازی},
	legend style={at={(0.5,-0.3)},
	anchor=north,legend columns=-1},
	xbar=5pt,
	ymax=270,ymin=80,
	      bar width=0.9em,
	height=6cm,width=12cm,
		        nodes near coords,
]

\addplot 
	coordinates {(0.24,100) (0.30,150) 
		(0.22,200) (0.24,250)};
		\addplot 
	coordinates {(0.24,100) (0.29,150)
		 (0.2,200) (0.23,250)};
\legend{ B\_M,S\_S}
\end{axis}
\end{tikzpicture}}
\caption{مقایسه کیفیت سرویس نقض شده در روش پیشنهادی و موردمقایسه با سیاست
\lr{MU}
}
\label{ch6}
\end{plot}

 در نمودار
(\ref{ch5})
و
(\ref{ch6})
 به بررسی کیفیت سرویس نقض شده پرداخته ایم کیفیت سرویس معمولا به فرم  SLA در محیط ابری شناخته می شود.تعداد ماشین های مجازی
۱۰۰
,
۱۵۰
,
۲۰۰
,
۲۵۰
  در نظر گرفته شده است.
همانطور که در نمودار
(\ref{ch5})
مشاهده می شود، ­، روش پیشنهادی نقض کیفیت سرویس کمتری در مقایسه با روش مورد مقایسه دارد .در روش S\_S با انتخاب مناسب ماشین مجازی برای مهاجرت و جای دهی مناسب ماشین های مجازی سعی کردیم از منابع ماشین های فیزیکی به طور موثر بهره مند شویم و احتمال وقوع نقض کیفیت سرویس  را بهبود بخشیم. 
در این فصل به بررسی و شبیه سازی روش پیشنهادی  و روش مورد مقایسه پرداختیم. در فصل بعد به نتیجه گیری  و کارهای آینده می‌پردازیم .
