\chapter{مقدمه}
\section{ مقدمه ای بر رایانش ابری}
 امروزه با پیشرفت روز افزون فناوری اطلاعات و افزایش برنامه های کاربردی ، 
 بی شک نیاز به محاسبات مسنجم و یکپارچه برای کاربران ضروری می باشد.
 همچنین با توجه به نیازهای کاربردی که کاربران دارند، 
 نیاز است که کاربران بتوانند کارهای پیچیده خود را بدون  اینکه 
 نیازی به داشتن سخت افزارها و نرم افزارهای گران قیمت داشته باشند، 
 از طریق اینترنت بتوانند انجام دهند. در واقع با این پردازش های سخت و سنگین ، 
 نیاز به پردازنده های متنوع و زیاد دارند تا بتوانند این کارهای پیچیده را با آنها انجام دهند.بنابراین استفاده از تکنولوژی مانند رایانش ابری که با توجه به نیاز کاربران، پردازش های محاسباتی آن ها آن ها را انجام دهد و نتایج را به آن ها 
 نمایش دهد، لازم می باشد.سیستم های رایانش ابری مراکز داده را با طراحی به صورت شبکه های مجازی، از نظر سخت افزار، پایگاه داده، نرم افزار و... توانمند کردند، 
 به طوری که کاربران بتوانند برنامه های کاربردی و موردنیاز خود را از هر جایی  با کمترین هزینه دریافت کنند.
  \cite{num2}
