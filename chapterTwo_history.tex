\فصل{مروری بر روش‌های انجام شده}\برچسب{فصل‌تاریخچه}
\قسمت {مروری بر روش‌های انجام شده}
  در مراکز داده ابری منابع مورد نیاز ماشین‌های مجازی ممکن است از ظرفیت سروری که روی آن‌ها میزبانی می‌شوند بیشتر شود. در نتیجه در مقیاس بزرگ این منابع نیاز به مدیریت خودکار دارند. انرژی مصرفی در محیط‌های ابری از دو جنبه مورد بررسی قرار  می‌گیرد جنبه اول مدیریت استاتیک انرژی که بیشتر مربوط به تجهیزات و سخت‌افزاری می‌باشد. جنبه دوم مدیریت پویای مصرف انرژی می‌باشد. در محیط ابری عمل  ترکیب پویای ماشین‌های مجازی با استفاده از مهاجرت ماشین‌های مجازی و خاموش ­کردن میزبان‌های فیزیکی بیکار باعث 
 بهینه‌شدن  مصرف منابع و کاهش مصرف انرژی می­‌شود­. با توجه به افزایش روزافزون محبوبیت سیستم‌های ابری، اگر انرژی ای که در منابع ارائه دهنده خدمات آن مصرف می‌شود کنترل نگردد، آنگاه هزینه ارائه سرویس‌های آن‌ها  افزایش می‌یابد و در پی آن روی هزینه پرداختی سرویس گیرندگان تأثیر خواهد گذاشت. مسئله مهمتر اینکه این مسئله سهم زیادی در افزایش آلودگی محیط زیست خواهد داشت. لذا کشف راهکارهای بهره وری انرژی بسیار حیاتی است. در این فصل قصد داریم به بررسی روش‌های انجام شده در زمینه مدیریت ماشین‌های مجازی، بهبود کیفیت سرویس و کاهش مصرف انرژی بپردازیم.
 
 \قسمت{روش چوی\پانویس{\lr{Choi}}}

در این مقاله
\cite{num8}
یک مرکز داده که در آن ارائه‌دهنده خدمات، ماشین‌های مجازی را روی میزبان‌های فیزیکی برای مشترکان خود برای محاسبات در شکل تقاضا است، تامین می‌کنند. برای مرکز داده ابری، یک الگوریتم ترکیب کار مبتنی بر دسته بندی کار (به عنوان مثال محاسباتی و داده‌ای) و استفاده منابع (مثل 
 \lr{CPU}
  و
   \lr{RAM}
   ) پیشنهاد شده است. علاوه بر این، یک الگوریتم ترکیب ماشین مجازی برای تعادل زمان اجرای کار و مصرف انرژی بدون نقض توافق نامه سطح خدمات
 \پانویس{
 	\lr{service level agreement}}
\lr{    (SLA)}
     طراحی شده است. برخلاف تحقیقات موجود بر روی ترکیب ماشین‌های مجازی یا زمانبندی که از طرح‌های آستانه تک استفاده می‌کنند، در این مقاله بر روی  طرح دو آستانه (بالا و پایین) که برای ترکیب ماشین مجازی استفاده می‌شود ، تمرکز شده است. به طور خاص، زمانی که یک میزبان با استفاده از منابع کمتر از آستانه پایین عمل می‌کند، همه ماشین‌های مجازی روی میزبان برای مهاجرت به میزبان‌های دیگر زمانبندی خواهند شد و پس از آن میزبان مربوطه خاموش خواهد شد، در حالیکه زمانی که یک میزبان با بهره وری منابع بالاتر از حد بالای آستانه عمل می‌کند، یک ماشین مجازی برای جلوگیری از  ۱۰۰ درصد استفاده از منابع مهاجرت داده خواهد شد. براساس ارزیابی تجربی با داده‌های واقعی، ثابت شده که دسته بندی کارها براساس الگوریتم ترکیب انرژی محور به کاهش قابل توجه انرژی بدون نقض 
      \lr{SLA}
       دست یافته است.
     \قسمت{روش چین\پانویس{
 \lr{Chein}}}
  مهاجرت ماشین‌های مجازی در محیط محاسبات ابری یک موضوع مهم برای حل خیلی از مسائل مانند توازن بار است که می‌تواند با  مهاجرت ماشین‌های مجازی از سرورهای بیش از حد بار شده و پربار و ترکیب سرورها که بار آن‌ها بعد از مهاجرت به دیگر سرورها می‌تواند پایین آید. در این مقاله
\cite{num9}
  یک الگوریتم مهاجرت ماشین مجازی مبتنی بر حداقل سازی مهاجرت در رایانش ابری برای بهبود بهره‌وری و پاسخ نیازها برای کاربر و محدودیت در نقض سطح کیفیت سرویس که به فرم 
   \lr{SLA}
    شناخته می‌شود، پیشنهاد شده است.نتایج آزمایشات موثر بودن الگوریتم پیشنهاد شده 
    را در مقایسه با الگوریتم‌های موجود نشان می‌دهد.اثر بخشی این تکنیک‌ها به حل خیلی از مسائل مثل موازنه بار ، حفظ سیستم  و غیره به منظور افزایش کارایی با استفاده از سیستم‌های ابری  و همچنین کیفیت خدمات به  مشتریان کمک می‌کند. در این مقاله یک الگوریتم تصمیم گیری کارامد مهاجرت ماشین مجازی در محیط ابری برای حل مسائل بالا ارائه شده است. 
    \قسمت{روش باسکار\پانویس{
\lr{Bhaskar}}}
     با رشد اخیر رایانش ابری، چالش بزرگ ارائه دهندگان سرویس مساله طراحی استراتژی موثری برای مدیریت منابع اشتراکی با برنامه‌های متفاوت است. مکانیزم مدیریت منابع باید اشتراک گذاری موثری از منابع را برای ماشین‌های مجازی با تضمین بهره برداری بهینه از منابع میزبان‌های فیزیکی در دسترس انجام دهد. مکانیزم مدیریت منابع به کاربران ابر و همچنین ارائه دهندگان خدمات اجازه می‌دهد که استفاده موثری از منابع در دسترس خود داشته باشند. این مقاله
\cite{num10}
     برنامه ای از مدل مجموعه راف برای فراهم کردن ماشین‌های مجازی پیشنهاد داده است.روش پیشنهاد شده از مشخصات/ دانش براساس روش‌های کاهش استفاده می‌کند.این روش قوانین را برای کاهش ویژگی‌های غیرضروری برای ماشین‌های مجازی تولید می‌کند.این قوانین به مدیریت ماشین‌های مجازی برای انتخاب موثر ماشین مجازی کمک می‌کند. این مقاله مشکلات تامین ماشین مجازی مورد تقاضا را مورد بررسی قرار داده است.تکنیک کاهش مبتنی بر دانش برای مساله تامین ماشین مجازی براساس منابع موجود را در نظر گرفته است. روش پیشنهاد شده قوانینی برای تصمیمات موثر در انتخاب و نگاشت برنامه‌ها به ماشین‌های مجازی برای مدیریت ماشین‌های مجازی تولید می‌کند.
     \قسمت{گودرزی\پانویس{
 \lr{Goudarzi} }}
در این کار 
\cite{num11}
، 
یک توافق‌نامه سطح خدمات 
\lr{(SLA)}
 مبتنی بر روش مدیریت منابع برای مراکز داده ابری ارائه شده است، که انرژی سرورهای موجود، محدودیت اوج انرژی و مصرف توان خنک کننده‌ها را در نظر گرفته است. هدف این مدیر منابع به حداقل رساندن هزینه‌های عملیاتی مراکز داده است. ساختار سلسله مراتبی روش پیشنهاد شده مدیریت منابع را مقیاس پذیر می‌سازد. روش مدیریت منابع پیشنهاد شده به طور همزمان سرور و مصرف توان خنک کننده‌ها را در نظر می‌گیرد و پیچیدگی تصمیم گیری در مدیریت منابع و
   \lr{SLA}
   را در سیستم‌های رایانش ابری تضمین می‌کند. در نظر گرفتن
\lr{    SLA}
    و حالت مراکز داده در شناسایی مقدار منابع مورد نیاز برای تخصیص به برنامه‌ها باعث کاهش قابل توجهی در هزینه‌های عملیاتی مراکز داده شده است. اثربخشی طرح مدیریت پیشنهاد شده در مقایسه با کارهای قبلی با استفاده از یک ابزار شبیه سازی جامع نشان داده شده است. الگوریتم‌های مدیریت منابع پیشنهاد شده هزینه‌های عملیاتی مراکز داده را حدود ۴۰ درصد کاهش داده در حالی‌که
     \lr{SLA}
     حفظ شده است و همچنین کاهش زمان اجرای الگوریتم‌های مدیریت تا  ۸۶ درصد با توجه به روش مدیریت متمرکز را بیان می‌کند .در این مقاله یک ساختار سلسله مراتبی مدیریت منابع برای سیستم ابری پیشنهاد شده است. ساختار ارائه شده مقیاس پذیری و کارایی بالایی را در مقایسه با یک ساختار متمرکز در کارهای قبلی نشان می‌دهد. علاوه بر انعطاف پذیری مبتنی بر
\lr{      SLA }
      با توجه به ویژگی ماشین‌های مجازی برای مساله مدیریت منابع، که  یک فاکتور مهم برای عملکرد بالاتر روش در مقایسه با روش‌های قبلی است. علاوه بر این، از دست دادن کارایی روش غیرمتمرکز با توجه به نسخه متمرکز شده الگوریتم کمتر از ۲ درصد ۲۷ بار زمان اجرای کوتاهتری داشته است. نتایج الگوریتم پیشنهاد شده در تناسب انرژی بالاتر در کل مراکز داده، نقض
       \lr{SLA} 
      و هزینه مهاجرت کمتر و بهره وری سیستم‌های خنک کننده بالاتری را نتیجه شده است. ساختار مدیریت پیشنهاد شده برای مهاجرت ماشین‌های مجازی محلی و تنظیم تخصیص منابع برای جلوگیری از افزایش دما، اوج توان و شرایط
       \lr{SLA} 
      ضروری است.
      \قسمت{روش اسماعیل\پانویس{
  \lr{Ismaeel}      }}
به منظور اجرای بهینه ترکیب ماشین‌های مجازی تحت محدودیت‌های کیفیت سرویس
\lr{(QoS) }
مبتنی بر مصرف انرژی در مراکز داده ابری که حاوی منابع فیزیکی ناهمگن است، باید یک چارچوب که ترکیبی از بسیاری از الگوریتم‌های زیر سیستمی‌ می‌باشد که شامل  پیش‌بینی انتخاب، قرار دادن، و غیره است ایجاد شود. چندین استراتژی به منظور حداقل رساندن مصرف انرژی در محیط ابری می‌تواند استفاده شود، اما مهمتر از آن این است که به حداقل رساندن از طریق خاموش کردن میزبان انتخاب شده کم بار بعد از جابجایی همه ماشین‌های مجازی روی سرور انتخاب شده انجام می‌شود. پیش بینی منابع مورد نیاز در یک دوره زمانی معین درحال حاضر اولین و مهمترین گام در تامین پویا برای براورد انتظارات
\lr{ QoS}
 در بارکاری‌های متغییر می‌باشد. به عبارت دیگر، در این مقاله 
\cite{num12}
 از الگوهای استفاده شده قبلی برای برآورد بارکاری درخواست شده برای آینده ماشین مجازی در مراکز داده  استفاده شده است. اولین گام در فرایند پیش بینی چارچوب مصرف انرژی به دسته داده‌های تاریخی(مهم) است. در این مقاله، یک دسته برای هر دو کاربر و درخواست‌های ماشین مجازی پیشنهاد شده است. بررسی گوگل واقعی که از ویژگی‌های بیش از ۲۵ میلیون کار جمع آوری شده بیش از یک دوره ۲۹ روزه به عنوان مثال در این مقاله استفاده شده است. نظارت باید برای جمع آوری داده از سطوح متفاوت از زیرساخت کل محاسبات(مثل ماشین مجازی، شبکه و ذخیره سازی ) و منابع نرم افزاری (مثل وب سرور ، دیتابیس سرور و برنامه ماشین مجازی) با استفاده از ابزاری مثل‌ اپن‌استک استفاده شود. انرژی مصرف شده با هر بخش از سخت افزار در مراکز داده می‌تواند با استفاده از ابزاری مثل مدیر زیرساخت مراکز داده 
\lr{ (DCIM)}
  نظارت شود. روش ارائه شده در این مقاله برای پیش بینی ماشین مجازی دسته کاربر و دسته ماشین مجازی برای دست یافتن به پیش‌بینی بهتر مصرف انرژی مراکز داده ابری ترکیب شده است. الگوریتم فازی
\lr{   c-means}
   نتایج بهتری را از روش مبتنی بر
\lr{    k-means }
   برای هر دو دسته، دسته کاربر و ماشین مجازی برای تعداد کمی‌ از دسته‌ها که بسیار مهم در کاهش تعداد ورودی در یک سیستم پیش بینی هستند نشان می‌دهد. صرف نظر از الگوریتم دسته بندی استفاده شده، دو هدف باید در نظر گرفته شود: کاهش خطا و حفظ سربار کم. به عبارت دیگر، اگرچه افزایش تعداد دسته‌ها در یک الگوریتم خطا را کاهش می‌دهد، این کار مساله پیش بینی و در نتیجه بهینه سازی مصرف انرژی را در مراکز داده ابری پیچیده می‌کند.
   \قسمت{روش راجو\پانویس{
\lr{Raju}}}
محاسبات ابری یک الگوی رایانشی توزیع‌شده در مقیاس بزرگ است که در آن یک استخر از منابع به صورت پویا مقیاس پذیر و مجازی مثل توان محاسباتی، ذخیره سازی، سیستم عامل و سرویس و تقاضا برای مشتریان خارجی از طریق اینترنت تحویل داده می‌شود. در زمانبندی محاسبات ابری فرایند تصمیم‌گیری برای تخصیص منابع در قالب ماشین‌های مجازی برای برنامه‌های درخواست شده می‌باشد. در این مقاله
\cite{num13}
دو مرحله زمانبندنی مهلت آگاه برای زمانبندی ماشین‌های مجازی برای برنامه‌های درخواست شده در محاسبات ابری از مشتریان دریافت شده پیشنهاد شده است. در این مدل هر برنامه به دو نوع ماشین مجازی برای تکمیل آن کار نیاز دارد. این مدل ماشین‌های مجازی را به عنوان منابع برای برنامه (جاب)‌های درخواست شده مبتنی بر زمان پردازش و زمانبندی برنامه‌ها با در نظر گرفتن مهلت با توجه به زمان پاسخ و زمان انتظار تخصیص می‌دهد. یک محیط شبیه سازی توسعه داده شده و ارزیابی شده برای ارزیابی این مدل با درنظر گرفتن معیارهای ارزیابی از میانگین زمان چرخش، میانگین زمان انتظار و نقض در مهلت زمانی که با الگوریتم‌های اول بهترین 
\lr{(FCFS)}
 و استراتژی زمانبندی کوتاهترین اول 
\lr{ (SJF)}
  مقایسه شده است. این مدل معیارهای ارزیابی را با فاکتور ثابت در مقایسه با سایر روش‌های زمانبندی کاهش می‌دهد. زمانبندی 
  \lr{n}
   جاب روی دو نوع از ماشین‌های مجازی با استفاده از الگوریتم زمانبندی مهلت آگاه دو مرحله ای عملکرد بهتری را در مقایسه با دیگر روش‌های زمانبندی می‌دهد.نتایج تجربی نشان می‌دهد که الگوریتم زمانبندی دو مرحله ای مهلت آگاه زمان انتظار میانگین، زمان برگشت میانگین، نقض مهلت میانگین با توجه به زمان انتظار، میانگین نقض مهلت با توجه به زمان پاسخ به طور معقولی در مقایسه با روش‌های 
\lr{    FCFS }
   و
\lr{    SJF}
    و الگوریتم‌های زمانبندی دو مرحله ای کاهش می‌دهد. تعداد نقض مهلت جاب‌ها با توجه به زمان پاسخ و زمان انتظار با در نظر گرفتن فاکتور ثابت در الگوریتم دو مرحله ای مهلت آگاه در مقایسه با الگوریتم‌های قبلی کاهش یافته است.
    \قسمت{روش دوان\پانویس{
    \lr{Duan}}}
 یکی از چالش‌های موجود در زمینه سیستم‌های ابری، چگونگی کاهش مصرف انرژی با حفظ ظرفیت محاسباتی بالا است. روش‌های موجود اساساً برروی افزایش بهره‌برداری منابع تمرکز کرده‌اند . برنامه‌های کاربردی با منابع مورد نیاز متفاوتی برروی ماشین‌های مجازی اجرا می‌شوند که برروی کارایی سیستم و مصرف انرژی تأثیر می‌گذارند. همچنین ممکن است که اوج بار
\پانویس{\
\lr{Peak loads}
}
  لحظه‌ای منجر به این شود  که در سودمندی مصرف انرژی تاثیر بگذارد .در تحقیق دیگری
\cite{num14}
  الگوریتم زمانبندی جدیدی با نام
\lr{  PreAntPolicy}
  ارائه شده است که شامل مدل پیش‌بینی بر‌اساس مکانیزم‌های فرکتال
 \پانویس{
\lr{Fractal} 
}
  و
   زمانبندی براساس بهبود الگوریتم کلونی است.  محققین مقاله با استفاده از تحلیل‌های زیاد و آزمایشات شبیه‌سازی در بارکاری واقعی محاسبات کلاسترهای گوگل توانستند کارایی کار خود را در سودمندی مصرف انرژی و بهره‌وری منابع نشان دهند. علاوه بر این روش پیشنهادی محققین مقاله مدل ذخیره تأمین ظرفیت پویای مؤثری را برای برنامه‌های کاربردی با نیازهای منابع متفاوت در محیط محاسبات ناهمگن را پیشنهاد می‌کند که می‌تواند مصرف منابع سیستم و انرژی را کاهش دهد به‌طوری‌که زمانبندی مناسبی را در زمان اوج بار فراهم می‌کند.  در آزمایشات شبیه‌سازی خود از الگوریتم‌های زمانبندی اول بهترین حریصانه
   \پانویس{
\lr{Greedy First-Fit (FF)}   
} 
   ،
    نوبت چرخشی
    \پانویس{
\lr{Round-Robin (RR)}    
}
     (که معمولاً توسط برخی از محاسبات ابری استفاده می‌شود)  و حداقل توان مهاجرت استفاده کردند. نتایج شبیه‌سازی نشان می‌دهد که روش پیشنهادی مقاله در مقایسه با الگوریتم اول بهترین
۱۷٫۷۶\%
     و در مقایسه با الگوریتم نوبت چرخشی
۱۸٫۷۵\%
     کاهش در مصرف انرژی داشته‌است، در حالیکه از نقض کیفیت سرویس درخواست شده تا جای ممکن جلوگیری شده است.
     \قسمت{روش پاتل\پانویس{
 \lr{Patel}    
 }}
یکی از چالش‌های مهم در سیستم‌های ابری، تخصیص منابع است. در تحقیق دیگری 
\cite{num15}
الگوریتمی‌به نام  بهترین کاهش اصلاح شده\پانویس{
\lr{Modified best fit decreasing  } 
} 
 به صورت الگوریتم انرژی محور
 \lr{EABFD}
 پیشنهاد شده است.روش 
 \lr{EABFD} 
 در ابتدا دو صف از میزبان‌های فیزیکی کم بار و خالی را تشکیل می‌دهد. صف میزبان‌های فیزیکی خالی و صف میزبان‌های کم بار را در ابتدا با هدف بهبود تخصیص ماشین‌های مجازی مقداردهی اولیه می‌کند. طبق این الگوریتم، همه ماشین‌های مجازی براساس کاهش بهره وری از پردازنده آنها  مرتب می‌شوند. سپس این الگوریتم ، بهترین میزبان فیزیکی را در میان همه میزبان‌های کم بار و خالی پیدا می‌کند. برای این منظور، در ابتدا، میزبان‌های کم بار را بررسی می‌کند ،در نهایت ،  اگر در میان همه میزبان‌های کم بار، میزبان فیزیکی مورد نیاز را  پیدا نکند، این الگوریتم یک میزبان از میزبان‌های خالی لیست برای تخصیص ماشین مجازی روی آن را روشن می‌کند. این الگوریتم تلاش دارد تعداد میزبان‌های روشن را به منظور کاهش مصرف انرژی حداقل کند.در این مقاله صرفه جویی در میزان انرژی با ترکیب موثر ماشین‌های مجازی انجام می‌شود. 
