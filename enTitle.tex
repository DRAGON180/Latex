% !TeX root=main.tex
% در این فایل، عنوان پایان‌نامه، مشخصات خود و چکیده پایان‌نامه را به انگلیسی، وارد کنید.

%%%%%%%%%%%%%%%%%%%%%%%%%%%%%%%%%%%%
\baselineskip=.6cm
\begin{latin}
\latinuniversity{Islamic Azad University }
\latinbranch{Sirjan}
\latinthesistype{M .Sc Thesis}
\latinfaculty{ Engineering }
\latinsubject{Computer Engineering }
\latinfield{Software Engeenring}
\latintitle{A new method to assign virtual machines to avoid overloading the physical host to improve the quality of service in the cloud data center }
\firstlatinsupervisor{Dr Mohammad Sadegh Hajmohammadi}
%\secondlatinsupervisor{Second Supervisor}
%\firstlatinadvisor{First Advisor}
%\secondlatinadvisor{Second Advisor}
\latinname{Molaee}
\latinsurname{Saeb}
\latinthesisdate{Summer 2017}
\latinkeywords{ Cloud computing, Allocation, Overload, Control of migration, Quality of Service, Energy Consumption.
}
\en-abstract
{Today, with the advent of information technology and the rise of applications, there is no doubt a need for an integrated calculation for users. Therefore, it is necessary to use technology such as a computer that performs their processing according to the needs of the users and displays the results to them. At the moment, there are a variety of cloud computing challenges. Effective use, allocation and management of resources to improve the quality of service and energy efficiency is one of the major challenges in cloud systems. The thesis focuses on choosing the proper destination for hosting virtual machines and controlling the migration of migrating virtual machines. Our goal is to automate the assignment of virtual machines to physical hosts so as to prevent the overload of physical hosts and extra migration. Experimental results compared to the base method show that with proper allocation and control in migration, we can improve the quality of service, while avoiding increasing energy consumption.
}
\latinfirstPage
\end{latin}
