% !TeX root=thesis.tex
% در این فایل، عنوان پایان‌نامه، مشخصات خود، متن تقدیمی‌، ستایش، سپاس‌گزاری و چکیده پایان‌نامه را به فارسی، وارد کنید.
% توجه داشته باشید که جدول حاوی مشخصات پروژه/پایان‌نامه/رساله و همچنین، مشخصات داخل آن، به طور خودکار، درج می‌شود.
%%%%%%%%%%%%%%%%%%%%%%%%%%%%%%%%%%%%
% دانشگاه خود را وارد کنید
\university{دانشگاه آزاد اسلامی سیرجان}
% دانشکده، آموزشکده و یا پژوهشکده  خود را وارد کنید
\faculty{دانشکده فنی و مهندسی}
% گروه آموزشی خود را وارد کنید
\department{گروه نرم‌افزار}
% گروه آموزشی خود را وارد کنید
\subject{مهندسی کامپیوتر}
% گرایش خود را وارد کنید
\field{نرم‌افزار}
% عنوان پایان‌نامه را وارد کنید
\title{بیان روشی به منظور تخصیص ماشین های مجازی برای
	جلوگیری از سربار شدن میزبان های فیزیکی با هدف بهبود
	کیفیت سرویس در مراکز داده ابری}
% نام استاد(ان) راهنما را وارد کنید
\firstsupervisor{دکترمحمدصادق حاج‌محمدی}
%\secondsupervisor{استاد راهنمای دوم}
% نام استاد(دان) مشاور را وارد کنید. چنانچه استاد مشاور ندارید، دستور پایین را غیرفعال کنید.
%\firstadvisor{استاد مشاور اول}
%\secondadvisor{استاد مشاور دوم}
% نام دانشجو را وارد کنید
\name{صائب}
% نام خانوادگی دانشجو را وارد کنید
\surname{ملایی ندیکی}
% شماره دانشجویی دانشجو را وارد کنید
\studentID{۹۳۰۵۹۴۱۳۱}
% تاریخ پایان‌نامه را وارد کنید
\thesisdate{تابستان ۱۳۹۶}
% به صورت پیش‌فرض برای پایان‌نامه‌های کارشناسی تا دکترا به ترتیب از عبارات «پروژه»، «پایان‌نامه» و »رساله» استفاده می‌شود؛ اگر  نمی‌پسندید هر عنوانی را که مایلید در دستور زیر قرار داده و آنرا از حالت توضیح خارج کنید.
\projectLabel{پایان‌نامه}

% به صورت پیش‌فرض برای عناوین مقاطع تحصیلی کارشناسی تا دکترا به ترتیب از عبارات «کارشناسی»، «کارشناسی ارشد» و »دکترا» استفاده می‌شود؛ اگر  نمی‌پسندید هر عنوانی را که مایلید در دستور زیر قرار داده و آنرا از حالت توضیح خارج کنید.
%\degree{کارشناسی ارشد}

\firstPage
\pagenumbering{harfi}
\besmPage
\davaranPage

%\vspace{.5cm}
% در این قسمت اسامی اساتید راهنما، مشاور و داور باید به صورت دستی وارد شوند
%\renewcommand{\arraystretch}{1.2}
\begin{center}
		\centering
\begin{tabular}{| p{8mm} | p{18mm} | p{36mm} |p{35mm}|p{.2\textwidth}|c|}

\hline
ردیف	& سمت & نام و نام خانوادگی &	دانشگاه یا مؤسسه &	امضـــــــــــــا\\
\hline
۱  &	استاد راهنما				 & دکتر \newline محمدصادق حاج‌محمدی&  دانشگاه \newlineآزاد اسلامی سیرجان &  \\
\hline
۲ &     استاد مدعو				 &  & دانشگاه \newline  & \\
\hline
\end{tabular}
\end{center}

\esalatPage
\mojavezPage


% چنانچه مایل به چاپ صفحات «تقدیم»، «نیایش» و «سپاس‌گزاری» در خروجی نیستید، خط‌های زیر را با گذاشتن ٪  در ابتدای آنها غیرفعال کنید.
 % پایان‌نامه خود را تقدیم کنید!

 \newpage
\thispagestyle{empty}
{\Large تقدیم به:}\\
\begin{flushleft}
{\huge
الکساندرا الباکیان\\
\vspace{7mm}
که با شعار "برای حذف همه موانع از راه علم" \\
\lr{To remove all barriers in the way of science}\\
 ۵۰ میلیون مقاله را مجانی در اختیار دنیا قرار داد\\

\vspace{7mm}

}
\end{flushleft}


% سپاس‌گزاری
\begin{acknowledgementpage}

در آغاز وظیفه‌  خود  می‌دانم از زحمات بی‌دریغ استاد  راهنمای خود،  جناب آقای دکترجاج‌محمدی صمیمانه تشکر و  قدردانی کنم  که قطعاً بدون راهنمایی‌های ارزنده‌  ایشان، این مجموعه  به انجام  نمی‌رسید.

همچنین لازم می‌دانم از فعالان حوزه نرم‌افزار آزاد که بدون هیچ چشم‌داشتی پاسخ همه پرسش‌های علاقه‌مندان به این حوزه را می‌دهند و منابع فعالیت‌های خود را در اختیار همه‌گان می‌گذارند.

در پایان از پدیدآورندگان بسته زی‌پرشین، مخصوصاً جناب آقای  وفا خلیقی، که این پایان‌نامه با استفاده از این بسته، آماده شده است و همه دوستانمان در گروه پارسی‌لاتک کمال قدردانی را داشته باشم.

% با استفاده از دستور زیر، امضای شما، به طور خودکار، درج می‌شود.
\signature 
\end{acknowledgementpage}
%%%%%%%%%%%%%%%%%%%%%%%%%%%%%%%%%%%%
% کلمات کلیدی پایان‌نامه را وارد کنید
\keywords{رایانش ابری، تخصیص، سرریز شدن، کنترل مهاجرت، کیفیت سرویس، مصرف انرژی.}
%چکیده پایان‌نامه را وارد کنید، برای ایجاد پاراگراف جدید از \\ استفاده کنید. اگر خط خالی دشته باشید، خطا خواهید گرفت.
\fa-abstract{
اﻣﺮوزه ﺑﺎ ﭘﯿﺸﺮﻓﺖ روز اﻓﺰون ﻓﻨﺎوری اﻃﻼﻋﺎت و  اﻓﺰاﯾﺶ ﺑﺮﻧﺎﻣﻪ ﻫﺎی ﮐﺎرﺑﺮدی، ﺑﯽ شک ﻧﯿﺎز ﺑﻪ ﻣﺤﺎﺳﺒﺎت یکپارچه ﺑﺮای ﮐﺎرﺑﺮان ﺿﺮوری می ﺑﺎﺷﺪ. ﺑﻨﺎﺑﺮاﯾﻦ اﺳﺘﻔﺎده از ﺗﮑﻨﻮﻟﻮژی ﻣﺎﻧﻨﺪ راﯾﺎﻧﺶاﺑﺮی ﮐﻪ ﺑﺎ ﺗﻮﺟﻪ ﺑﻪ ﻧﯿﺎز ﮐﺎرﺑﺮان، ﭘﺮدازش‌ﻫﺎی ﻣﺤﺎﺳﺒﺎتی آن‌ﻫﺎ را اﻧﺠﺎم دﻫﺪ و ﻧﺘﺎﯾﺞ را ﺑﻪ آن‌ﻫﺎ ﻧﻤﺎﯾﺶدﻫﺪ، ﻻزم می‌باشد. درﺣﺎل ﺣﺎﺿﺮ ﭼﺎﻟﺶﻫﺎی ﻣﺘﻨﻮعی در زﻣﯿﻨﻪ راﯾﺎﻧﺶ اﺑﺮی ﻣﻄﺮح اﺳﺖ. استفاده مؤثر، ﺗﺨﺼﯿﺺ و مدیریت منابع به‌منظور بهبود کیفیت سرویس و بهره­وری انرژی یکی از ازﭼﺎﻟﺶﻫﺎی ﻣﻬﻢ در سیستم‌های ابری است.
در این پایان نامه تمرکز بر روی انتخاب مقصد مناسب برای میزبانی ماشین های مجازی و همچنین اعمال کنترلی در مهاجرت ماشین‌های مجازی مهاجر می‌باشد. هدف ما از انجام این کار این است ﮐﻪ ﺗﺨﺼﯿﺺ ﻣﺎﺷﯿﻦﻫﺎی ﻣﺠﺎزی ﺑﻪ ﻣﯿﺰﺑﺎنﻫﺎی ﻓﯿﺰﯾکی را ﺑﻪ ﭼﻪ ﻧﺤﻮی اﻧﺠﺎم دﻫﯿﻢ ﮐﻪ ﺗﺎ ﺟﺎی ﻣمکن از ﺳﺮﯾﺰﺷﺪن ﻣﯿﺰﺑﺎنﻫﺎی ﻓﯿﺰﯾکی و مهاجرت اضافی جلوگیری ﮐﻨﯿﻢ. نتایج تجربی آزمایشات در مقایسه با روش‌ پایه نشان دهنده این است که با تخصیص مناسب و اعمال کنترلی در مهاجرت می‌‌توانیم در بهبود کیفیت سرویس تأثیرگذار باشیم درحالیکه از افزایش مصرف انرژی جلوگیری می‌کنیم.
}



\abstractPage

\newpage\clearpage
