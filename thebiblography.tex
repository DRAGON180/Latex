% !TeX root=thesis.tex

\begin{thebibliography}{99}

\begin{LTRbibitems}
\resetlatinfont
	\bibitem{define}
	Mell, P., \& Grance, T. (2011). The NIST definition of cloud computing.
	\bibitem{num2}
	Rittinghouse, J. W., \& Ransome, J. F. (2016). Cloud computing: implementation, management, and security. CRC press.
	\bibitem{num3}
	Höfer, C. N., \& Karagiannis, G. (2011). Cloud computing services: taxonomy and comparison. Journal of Internet Services and Applications, 2(2), 81-94.
		\bibitem{num4}
	Da Cunha Rodrigues, G., Calheiros, R. N., Guimaraes, V. T., Santos, G. L. D., de Carvalho, M. B., Granville, L. Z., ... \& Buyya, R. (2016, April). Monitoring of cloud computing environments: concepts, solutions, trends, and future directions. In Proceedings of the 31st Annual ACM Symposium on Applied Computing (pp. 378-383). ACM.
	\bibitem{num5}
	García-Valls, M., Cucinotta, T., \& Lu, C. (2014). Challenges in real-time virtualization and predictable cloud computing. Journal of Systems Architecture, 60(9), 726-740.
	\bibitem{num6}
	Yang, M., Li, Y., Jin, D., Zeng, L., Wu, X., \& Vasilakos, A. V. (2015). Software-defined and virtualized future mobile and wireless networks: A survey. Mobile Networks and Applications, 20(1), 4-18.
	\bibitem{num7}
	 Subramanian, M., Bodge, A., \& Pattabhi, R. (2016). U.S. Patent No. 20,160,019,265. Washington, DC: U.S. Patent and Trademark Office.
	 \bibitem{num8}
	 Choi, H., Lim, J., Yu, H., \& Lee, E. (2016). Task Classification Based Energy-Aware Consolidation in Clouds. Scientific Programming, 2016.
	 \bibitem{num9}
	 Chien, N. K., Dong, V. S. G., Son, N. H., \& Loc, H. D. (2016, March). An Efficient Virtual Machine Migration Algorithm Based on Minimization of Migration in Cloud Computing. In International Conference on Nature of Computation and Communication (pp. 62-71). Springer International Publishing.
	 \bibitem{num10}
	 Bhaskar, R., \& Shylaja, B. S. (2016). KNOWLEDGE BASED REDUCTION TECHNIQUE FOR VIRTUAL MACHINE PROVISIONING IN CLOUD COMPUTING. International Journal of Computer Science and Information Security, 14(7), 472.
	 \bibitem{num11}
	 Goudarzi, H., \& Pedram, M. (2016). Hierarchical SLA-driven resource management for peak power-aware and energy-efficient operation of a cloud datacenter.
	 \bibitem{num12}
	  Ismaeel, S., Miri, A., \& Al-Khazraji, A. (2016, March). Energy-consumption clustering in cloud data centre. In 2016 3rd MEC International Conference on Big Data and Smart City (ICBDSC) (pp. 1-6). IEEE.
	  \bibitem{num13}
	  Raju, I. R. K., Varma, P. S., Sundari, M. R., \& Moses, G. J. (2016). Deadline aware two stage scheduling algorithm in cloud computing. Indian Journal of Science and Technology, 9(4).
	  \bibitem{num14}
	  Duan, H., Chen, C., Min, G., \& Wu, Y. (2016). Energy-aware scheduling of virtual machines in heterogeneous cloud computing systems. Future Generation Computer Systems.
	  \bibitem{num15}
	  Patel, R., Patel, H., \& Patel, S. (2015). Quality of Service Based Efficient Resource Allocation in Cloud Computing, International Journal For Technological Research In Engineering Volume 2, Issue 9.
	  \bibitem{num16}
	  Beloglazov, A., \& Buyya, R. (2013). “Optimal online deterministic algorithms and adaptive heuristics for energy and performance efficient dynamic consolidation of virtual machines in cloud data centers.” Concurrency and Computation: Practice and Experience, 24(13), 1397-1420.
	  \bibitem{num17}
	  Tani, H. G., \& El Amrani, C. (2016). Cloud Computing CPU Allocation and Scheduling Algorithms using CloudSim Simulator. International Journal of Electrical and Computer Engineering, 6(4), 1866.
	\end{LTRbibitems}
\end{thebibliography}
